%----------------------------------------------------------------------------------------
%	PACKAGES AND THEMES
%----------------------------------------------------------------------------------------

\documentclass{beamer}
\usetheme{umbc1}%此处可以选umbc2~4
\usepackage{biblatex}
\addbibresource{references.bib}

%定义block颜色
\setbeamertemplate{blocks}[rounded][shadow=true]
\setbeamercolor{block title}{use=structure,fg=structure.fg,bg=structure.fg!20!bg}
\setbeamercolor{block body}{parent=normal text,use=block title,bg=block title.bg!50!bg}
\setbeamercolor*{block title example}{fg=black!40!green,bg=white!80!green}
\setbeamercolor*{block body example}{parent=normal text,use=block title example,bg=block title example.bg!50!bg}
\setbeamercolor*{block title alerted}{fg=black!40!red,bg=white!80!red}
\setbeamercolor*{block body alerted}{parent=normal text,use=block title alerted,bg=block title alerted.bg!50!bg}
% 设定中文字体
\usepackage[BoldFont,SlantFont,CJKchecksingle,CJKnumber]{xeCJK}
\usefonttheme[onlymath]{serif}
\setCJKmainfont[BoldFont={SimHei},ItalicFont={FangSong}]{SimSun}
\setCJKsansfont{SimSun}
\setCJKmonofont{FangSong}
\defaultfontfeatures{Mapping=tex-text}
\usepackage{xunicode}
\usepackage{xltxtra}
\XeTeXlinebreaklocale "zh"
\XeTeXlinebreakskip = 0pt plus 1pt minus 0.1pt
 
\usepackage{colortbl,xcolor}
\usepackage{hyperref}
\hypersetup{xetex,bookmarksnumbered=true,bookmarksopen=true,pdfborder=1,breaklinks, colorlinks, linkcolor=, urlcolor=structure.fg}
 
\usepackage{graphicx} % Allows including images
\usepackage{booktabs} % Allows the use of \toprule, \midrule and \bottomrule in tables
\usepackage{mathtools} %数学公式中case情况
\usepackage{tikz}% tikz做图
\usetikzlibrary{snakes}
\usepackage{minted}%code highlighting

%设定目录
\AtBeginSection[]%自动加入目录
{
\begin{frame}<beamer>
\frametitle{目录}
\tableofcontents[currentsection]
\end{frame}
}
\setbeamertemplate{section in toc}[sections numbered]
\setbeamertemplate{subsection in toc}[subsections numbered]
\defbeamertemplate{subsection page}{mine}[1][]{%
  \begin{centering}
    {\usebeamerfont{subsection name}\usebeamercolor[fg]{subsection name}#1}
    \vskip1em\par
    \begin{beamercolorbox}[sep=8pt,center,#1]{part title}
      \usebeamerfont{subsection title}\thesection.\thesubsection~\insertsubsection\par
    \end{beamercolorbox}
  \end{centering}
}
\setbeamertemplate{subsection page}[mine]
\AtBeginSubsection{\frame{\subsectionpage}}%自动加入标题页面

%引用文献高亮
\makeatletter
\let\@mycite\@cite
\def\@cite#1#2{{\hypersetup{linkcolor=blue!60!black}[{#1\if@tempswa , #2\fi}]}}
\makeatother

%中文图、表
\renewcommand{\figurename}{图}
\renewcommand{\tablename}{表}
\setbeamertemplate{caption}[numbered] %图表编号

%mindmap
\usetikzlibrary[topaths,mindmap,backgrounds]
\usetikzlibrary{chains,decorations.pathmorphing,positioning,fit}
\usetikzlibrary{decorations.shapes,calc,}
\usetikzlibrary{decorations.text,matrix}
\usetikzlibrary{arrows,shapes.geometric,shapes.symbols,scopes}
% A counter, since TikZ is not clever enough (yet) to handle
% arbitrary angle systems.
\newcount\mycount
\tikzset{
    invisible/.style={opacity=0},
    visible on/.style={alt=#1{}{invisible}},
    alt/.code args={<#1>#2#3}{%
    \alt<#1>{\pgfkeysalso{#2}}{\pgfkeysalso{#3}}
  },
}

\tikzset{
  orp/.style={
    overlay,
    remember picture,
  },
}
\newcommand<>\fadetitle[2][]{%
    \begin{tikzpicture}
      [
        orp,
        fade title fill/.style={
          fill=white,
          opacity=0.8,
        },
        #1
      ]
      \path[fade title fill] (current page.south west) +(-1cm, -1cm) rectangle ($ (current page.north east) + (1cm, -1cm) $);
      \node[above] at (current page) {\textbf{#2}};
    \end{tikzpicture}%
}
\tikzstyle{nodetype1pre}= [circle, fill=gray!60]
\tikzstyle{nodetype1}= [circle, fill=structure.fg!30]
\tikzstyle{unnodetype1}= [circle, fill=gray!60]
\tikzstyle{linktype1pre}= [-latex]
\tikzstyle{linktype1}= [-latex, draw=purple, line width=1.5pt]
\tikzstyle{dilinktype1pre}= [-]
\tikzstyle{dilinktype1}= [-, draw=purple, line width=1.5pt]
\tikzstyle{unlinktype1}= [-latex]
\tikzstyle{linktype2}= [-latex, snake=snake,line after snake=4mm, draw=structure.fg, line width=1.5pt]
\tikzstyle{every picture}+=[remember picture]

\newcommand{\point}[1]{\textbf{\usebeamercolor[fg]{structure}#1}} %定义高亮文本
\newcommand{\bcite}[1]{\textbf{\textcolor{red!80}{\textsuperscript{[}\footfullcite{#1}\textsuperscript{]}}}}
%----------------------------------------------------------------------------------------
%	TITLE PAGE
%----------------------------------------------------------------------------------------

\title[中文Beamer]{中文Beamer模板说明} 
\author[shortname]{author1 \inst{1} \and author2 \inst{2}}
\institute[shortinst]{\inst{1} affiliation for author1 \and %
                      \inst{2} affiliation for author2}
\date{某年某月某日} 

\begin{document}

\begin{frame}
\titlepage 
\end{frame}

\begin{frame}
\vspace*{12em}
\begin{itemize}
  \item  \tiny{本作品采用知识共享 署名-非商业性使用-相同方式共享 3.0 中国大陆 许可协议进行许可。要查看该许可协议,可访问 \url{http://creativecommons.org/licenses/by-nc-sa/3.0/cn/}}
\end{itemize}
\end{frame}

\begin{frame}
\frametitle{目录} 
\tableofcontents 
\end{frame}

%----------------------------------------------------------------------------------------
%	PRESENTATION SLIDES
%----------------------------------------------------------------------------------------
%------------------------------------------------
\section{文字部分} 

\subsection{中英文混合排版}
\begin{frame}
\frametitle{Paragraphs of Text}
中英文混合排版,中英文混合排版,中英文混合排版,中英文混合排版,中英文混合排版,中英文混合排版,中英文混合排版,中英文混合排版,中英文混合排版,\point{中英文混合排版},Sed iaculis dapibus gravida. Morbi sed tortor erat, nec interdum arcu. Sed id lorem lectus. Quisque viverra augue id sem ornare non aliquam nibh tristique. Aenean in ligula nisl. Nulla sed tellus ipsum. Donec vestibulum ligula non lorem vulputate fermentum accumsan neque mollis.
\end{frame}

%------------------------------------------------

\begin{frame}
\frametitle{Bullet Points}
\begin{itemize}
\item Lorem ipsum dolor sit amet, consectetur adipiscing elit
\item Aliquam blandit \tikz[baseline,inner sep=0] \node[anchor=base] (n1) {paths};faucibus nisi, sit amet dapibus enim tempus eu
\item Nulla commodo, erat quis gravida posuere, elit lacus lobortis est, quis porttitor odio mauris at libero
\item Nam cursus est eget velit posuere pellentesque
\item Vestibulum faucibus velit a augue condimentum quis \tikz[baseline,inner sep=0] \node[anchor=base] (n2) {convallis };nulla gravida
\end{itemize}
\pause
\tikz[overlay]\draw[thick,green,->] (n2) -- (n1);
\end{frame}

%------------------------------------------------

\begin{frame}
\frametitle{Blocks of Highlighted Text}
\begin{block}{普通框}
Lorem ipsum dolor sit amet, consectetur adipiscing elit. Integer lectus nisl, ultricies in feugiat rutrum, porttitor sit amet augue. Aliquam ut tortor mauris. Sed volutpat ante purus, quis accumsan dolor.
\end{block}

\begin{example}{举例框}
Pellentesque sed tellus purus. Class aptent taciti sociosqu ad litora torquent per conubia nostra, per inceptos himenaeos. Vestibulum quis magna at risus dictum tempor eu vitae velit.
\end{example}

\begin{alertblock}{警告框}
Suspendisse tincidunt sagittis gravida. Curabitur condimentum, enim sed venenatis rutrum, ipsum neque consectetur orci, sed blandit justo nisi ac lacus.
\end{alertblock}
\end{frame}

%------------------------------------------------

\begin{frame}
\frametitle{Multiple Columns}
\begin{columns}[c] % The "c" option specifies centered vertical alignment while the "t" option is used for top vertical alignment

\column{.45\textwidth} % Left column and width
\textbf{Heading}
\begin{enumerate}[<+->]
\item Statement
\item Explanation
\item Example
\end{enumerate}

\column{.5\textwidth} % Right column and width
Lorem ipsum dolor sit amet, consectetur adipiscing elit. Integer lectus nisl, ultricies in feugiat rutrum, porttitor sit amet augue. Aliquam ut tortor mauris. Sed volutpat ante purus, quis accumsan dolor.

\end{columns}
\end{frame}


\section{图、表、公式} 
\subsection{普通插图} %a subsection can be created just before a set of slides with a common theme to further break down your presentation into chunks
\begin{frame}
\frametitle{Figure}
Uncomment the {code} on this slide to include your own image from the same {directory} as the template .TeX file.
\begin{figure}
\includegraphics[width=0.2\linewidth]{pic/1.jpg}
\caption{example}
\end{figure}
\end{frame}

\subsection{tikz绘图} %a subsection can be created just before a set of slides with a common theme to further break down your presentation into chunks
\begin{frame}
\begin{tikzpicture}[scale=2,cap=round]
  % Local definitions
  \def\costhirty{0.8660256}

  % Colors
  \colorlet{anglecolor}{green!50!black}
  \colorlet{sincolor}{red}
  \colorlet{tancolor}{orange!80!black}
  \colorlet{coscolor}{blue}

  % Styles
  \tikzstyle{axes}=[]
  \tikzstyle{important line}=[very thick]
  \tikzstyle{information text}=[rounded corners,fill=red!10,inner sep=1ex]

  % The graphic
  \draw[style=help lines,step=0.5cm] (-1.4,-1.4) grid (1.4,1.4);

  \draw (0,0) circle (1cm);

  \begin{scope}[style=axes]
    \draw[->] (-1.5,0) -- (1.5,0) node[right] {$x$};
    \draw[->] (0,-1.5) -- (0,1.5) node[above] {$y$};

    \foreach \x/\xtext in {-1, -.5/-\frac{1}{2}, 1}
      \draw[xshift=\x cm] (0pt,1pt) -- (0pt,-1pt) node[below,fill=white]
            {$\xtext$};

    \foreach \y/\ytext in {-1, -.5/-\frac{1}{2}, .5/\frac{1}{2}, 1}
      \draw[yshift=\y cm] (1pt,0pt) -- (-1pt,0pt) node[left,fill=white]
            {$\ytext$};
  \end{scope}

  \filldraw[fill=green!20,draw=anglecolor] (0,0) -- (3mm,0pt) arc(0:30:3mm);
  \draw (15:2mm) node[anglecolor] {$\alpha$};

  \draw[style=important line,sincolor]
    (30:1cm) -- node[left=1pt,fill=white] {$\sin \alpha$} +(0,-.5);

  \draw[style=important line,coscolor]
    (0,0) -- node[below=2pt,fill=white] {$\cos \alpha$} (\costhirty,0);

  \draw[style=important line,tancolor] (1,0) --
    node [right=1pt,fill=white]
    {
      $\displaystyle \tan \alpha \color{black}=
      \frac{{\color{sincolor}\sin \alpha}}{\color{coscolor}\cos \alpha}$
    } (intersection of 0,0--30:1cm and 1,0--1,1) coordinate (t);

  \draw (0,0) -- (t);

  \end{tikzpicture}
  \frametitle{tikz picture sample}
\end{frame}

\begin{frame}
\frametitle{手绘图}
\tikzstyle{every picture}+=[remember picture]
\[  y =  \tikz[baseline]{\node[fill=blue!50,anchor=base] (t1){$a$};} x +
\tikz[baseline]{\node[fill=red!50,anchor=base ] (t2){$b$};}
\]
\begin{itemize}
\item \tikz\node [fill=blue!50,draw,circle] (n1) {};\ slope
\item \tikz\node [fill=red!50,draw,circle] (n2) {};\ y-intercept
\end{itemize}
\begin{tikzpicture}[overlay,>=latex]
\path[blue,->] (n1.north) edge [out= 60, in= 135] (t1.north west);
\path[red,->] (n2.south) edge [out=-70, in=-110] (t2.south);
\end{tikzpicture}
\end{frame}
%------------------------------------------------

%------------------------------------------------
\subsection{公式和列表} 

\begin{frame}
\frametitle{公式说明}
\begin{equation}
  \frac{\mathrm{d}\mathbf{\mathrm{x}}(t)}{\mathrm{d}t} = A\mathbf{\mathrm{x}}(t)+ B\mathbf{\mathrm{u}}(t)\text{。}
\label{eq:control}
\end{equation}

其中:
\begin{itemize}
\item 向量$\mathbf{\mathrm{x}}(t)$表示$N$个在$t$时
\item $A$表示$N$个度
\item $\mathbf{\mathrm{u}}$
\item $B$表示位点
\end{itemize}

\pause
\begin{block}{注意}
这是一个block。
\end{block}
\end{frame}

%------------------------------------------------

\begin{frame}
\frametitle{Table}
\begin{table}
\begin{tabular}{l l l}
\toprule
\textbf{Treatments} & \textbf{Response 1} & \textbf{Response 2}\\
\midrule
Treatment 1 & 0.0003262 & 0.562 \\
Treatment 2 & 0.0015681 & 0.910 \\
Treatment 3 & 0.0009271 & 0.296 \\
\bottomrule
\end{tabular}
\caption{Table caption}
\end{table}
\end{frame}


\subsection{定理、定义}

\begin{frame}
\frametitle{Theorem}
\begin{theorem}[勾股定理]
$c^2 = a^2 + b^2$
\end{theorem}
\end{frame}

\begin{frame}[fragile] % Need to use the fragile option when verbatim is used in the slide
  \frametitle{Verbatim}
  \begin{example}[Theorem Slide Code]
    \begin{verbatim}
    \begin{frame}
      \frametitle{Theorem}
      \begin{theorem}[勾股定理]
	$c^2 = a^2 + b^2$
      \end{theorem}
    \end{frame}
    \end{verbatim}
  \end{example}
\end{frame}

\subsection{代码高亮}

\begin{frame}[fragile]
  \frametitle{代码高亮}
C代码:
\begin{minted}[mathescape,
               linenos,     % displays the line numbers
               frame=leftline,
		baselinestretch=0.5]{csharp}
  /*
  这里可以显示公式:
  $\pi=\lim_{n\to\infty}\frac{P_n}{d}$ where $P$ is the perimeter
  of an $n$-sided regular polygon circumscribing a
  circle of diameter $d$.
  */
  const double pi = 3.1415926535
\end{minted}

python代码:
\begin{minted}[mathescape, linenos]{python}
# Returns $\sum_{i=1}ˆ{n}i$
# 多样注释格式和缩进,行码
def sum_from_one_to(n):
    r = range(1, n + 1)
    return sum(r)
\end{minted}
\end{frame}

\subsection{文献引用举例}
\begin{frame}
  \frametitle{宇宙大爆炸的定义}
$x^2+y^2=z^2$\bcite{bcite1}
\pause
\begin{definition}
  \point{宇宙大爆炸}:$(X_0,Y_0)$当且仅当$\forall \epsilon >0$\bcite{bcite2}。
\end{definition}
\end{frame}



%----------------------------------------

\subsection{左右分栏和图形动画} 

\begin{frame}
\begin{columns}[c] % The "c" option specifies centered vertical alignment while the "t" option is used for top vertical alignment
\column{.45\textwidth} % Left column and width
  \begin{figure}
    \begin{center}
    \tikzstyle{linktype1visible}= [visible on=<{4-}>]
    \tikzstyle{linktype1previsible}= [visible on=<{1-3}>]
    \tikzstyle{nodetype1visible}= [visible on=<{4-}>]
    \tikzstyle{nodetype1previsible}= [visible on=<{1-3}>]
    \tikzstyle{linktype2visible}= [visible on=<{5-}>]
    \tikzstyle{scalestyle}= [scale=0.7]
      \begin{tikzpicture}
	\node[scalestyle, visible on=<{1,2,3,4}>, nodetype1pre] (N-1) at (0, 1) {1};
	\node[scalestyle, visible on=<{5-}>, nodetype1] (N-1) at (0, 1) {1};
	\foreach \name/\x in {2/2, 3/3, 4/4, 5/5}
	{
	  \node[scalestyle, nodetype1pre, nodetype1previsible] (N-\name) at (0, \x) {$\name$};
	  \node[scalestyle, nodetype1, nodetype1visible] (N-\name) at (0, \x) {$\name$};
	}
	\foreach \name/\x/\y in {6/1/4, 7/2/5, 8/3/4, 9/2/3}
	{
	  \node[scalestyle, nodetype1pre, nodetype1previsible] (N-\name) at (\x, \y) {$\name$};
	  \node[scalestyle, nodetype1, nodetype1visible] (N-\name) at (\x, \y) {$\name$};
	}
	\foreach \name/\x/\y in {10/-1/2, 11/-2/2}
	{
	  \node[scalestyle, nodetype1pre, nodetype1previsible] (N-\name) at (\x, \y) {$\name$};
	  \node[scalestyle, nodetype1, nodetype1visible] (N-\name) at (\x, \y) {$\name$};
	}
	\foreach \from/\to in {1/2, 2/3, 3/4, 4/5, 6/7, 7/8, 8/9, 9/6}
	{
	  \draw[unlinktype1, linktype1previsible] (N-\from) -- (N-\to);
	}
	\draw[unlinktype1](N-4) -- (N-6);
	\draw[unlinktype1](N-2) -- (N-10);
	\path[unlinktype1] (N-5) edge [loop above] ();
	\draw[linktype1pre, linktype1previsible] (N-10) .. controls + (up:0.5cm) .. (N-11);
	\draw[linktype1pre, linktype1previsible] (N-11) .. controls + (down:0.5cm) .. (N-10);

	\foreach \from/\to in {1/2, 2/3, 3/4, 4/5, 6/7, 7/8, 8/9, 9/6}
	{
	  \draw[linktype1, linktype1visible] (N-\from) -- (N-\to);
	}
	\draw[linktype1, linktype1visible] (N-10) .. controls + (up:0.5cm) .. (N-11);
	\draw[linktype1, linktype1visible] (N-11) .. controls + (down:0.5cm) .. (N-10);
	\draw[linktype2, linktype2visible] (0,-1) --  (0,0.7);
      \end{tikzpicture}
    \end{center}
  \end{figure}
\column{.45\textwidth} % Left column and width
 \begin{itemize}
        \item<1-| alert@1>
	  第一条说明
        \item<4-| alert@4>
	  第二条说明
        \item<5-| alert@5>
	  第三条说明
        \item<8-| alert@8>
	  第四条说明
 \end{itemize}
 \visible<2>{\fadetitle{这是一个透明标题}}
\end{columns}
\visible<6>{
\fadetitle{\(\displaystyle
  A=\left( 
    \begin{array}{*{20}{rcccccccccl}}
      0&0&0&0&0&0&0&0&0&0&0\\
      a&0&0&0&0&0&0&0&0&0&0\\
      0&b&0&0&0&0&0&0&0&c&0\\
    \end{array}
  \right)
  \)
}
}
\end{frame}
%------------------------------------------------

\begin{frame}
\begin{figure}
  \begin{center}
    \tikzstyle{nodetype1visible}= [visible on=<{2-}>]
    \tikzstyle{nodetype1previsible}= [visible on=<{1}>]
    \tikzstyle{linktype2visible}= [visible on=<{3-}>]
    \tikzstyle{linktype1visible}= [visible on=<{2-}>]
    \tikzstyle{linktype1previsible}= [visible on=<{1}>]

    \begin{minipage}[b]{0.45\linewidth}
      \begin{tikzpicture}[scale=0.7]
	%-------------------------------------
	\node[unnodetype1,label=below left:$x_2$] (x2)  at (0,0) {};
	\node[unnodetype1,label=above right:$x_1$] (x1)  at (2,1) {};
	\node[nodetype1pre,nodetype1previsible,label=below right:$x_4$] (x4)  at (2,-1) {};
	\node[nodetype1,nodetype1visible,label=below right:$x_4$] (x4)  at (2,-1) {};
	\node[unnodetype1,label=below right:$x_3$] (x3)  at (4,0) {};
	%-------------------------------------
	\draw[-latex] (x1) -- (x2);
	\draw[-latex] (x1) -- (x3);
	\draw[linktype1pre,linktype1previsible] (x1) -- (x4);
	\draw[linktype1,linktype1visible] (x1) -- (x4);
	\draw[linktype2,linktype2visible] (2,3) node [right] {$u_1$} --  (x1);
	\draw[linktype2,linktype2visible] (4,2) node [right] {$u_3$}-- (x3);
	\draw[linktype2,linktype2visible] (0,2) node [left] {$u_2$}-- (x2);
	%-------------------------------------
    \end{tikzpicture}
    \end{minipage}
    \quad
    \pause
    \begin{minipage}[b]{0.45\linewidth}
    \begin{tikzpicture}[scale=0.7]
	%-------------------------------------
	\node[nodetype1pre,nodetype1previsible,label=below left:$x_2$] (x2)  at (0,0) {};
	\node[nodetype1,nodetype1visible,label=below left:$x_2$] (x2)  at (0,0) {};
	\node[unnodetype1,label=above right:$x_1$] (x1)  at (2,1) {};
	\node[unnodetype1,label=below left:$x_4$] (x4)  at (2,-1) {};
	\node[unnodetype1,label=below right:$x_3$] (x3)  at (4,0) {};
	%-------------------------------------
	\draw[-latex] (x1) -- (x4);
	\draw[-latex] (x1) -- (x3);
	\draw[linktype1pre,linktype1previsible] (x1) -- (x2);
	\draw[linktype1,linktype1visible] (x1) -- (x2);
	\draw[linktype2, linktype2visible] (2,3) node [right] {$u_1$} --  (x1);
	\draw[linktype2, linktype2visible] (4,2) node [right] {$u_3$}-- (x3);
	\draw[linktype2, linktype2visible] (4,-2) node [right] {$u_4$}-- (x4);
	%-------------------------------------
    \end{tikzpicture}
    \end{minipage}
    \begin{minipage}[b]{0.45\linewidth}
      \begin{tikzpicture}[scale=0.7]
	%-------------------------------------
	\node[unnodetype1,label=below left:$x_2$] (x2)  at (0,0) {};
	\node[unnodetype1,label=above right:$x_1$] (x1)  at (2,1) {};
	\node[unnodetype1,label=below left:$x_4$] (x4)  at (2,-1) {};
	\node[nodetype1pre,nodetype1previsible,label=below right:$x_3$] (x3)  at (4,0) {};
	\node[nodetype1,nodetype1visible,label=below right:$x_3$] (x3)  at (4,0) {};
	%-------------------------------------
	\draw[-latex] (x1) -- (x2);
	\draw[-latex] (x1) -- (x4);
	\draw[linktype1pre, linktype1previsible] (x1) -- (x3);
	\draw[linktype1, linktype1visible] (x1) -- (x3);
	\draw[linktype2,linktype2visible] (2,3) node [right] {$u_1$} --  (x1);
	\draw[linktype2,linktype2visible] (4,-2) node [right] {$u_4$}-- (x4);
	\draw[linktype2,linktype2visible] (0,2) node [left] {$u_2$}-- (x2);
	%-------------------------------------
	\node[unnodetype1,label=right:Original node] at (6.25,2) {};
	\visible<2->{\node[nodetype1,nodetype1visible,label=right:Matched node] at (6.25,1) {};}
	\draw[linktype1,linktype1visible] (5, 0) -- (6.5, 0) node [right] {Link type 1};
	\draw[linktype2,line after snake=2mm, linktype2visible] (5,-1) --(6.5,-1) node [right] {Link type 2};
    \end{tikzpicture}
    \end{minipage}
  \end{center}
\end{figure}
\end{frame}

%------------------------------------------------

\section{思维导图} 
\begin{frame}
  \frametitle{思维导图,总结,致谢}
  \setbeamercovered{invisible}
\begin{tikzpicture}[scale=0.6, transform shape]
\bf
\centering
  \path[mindmap,concept color=structure.fg,text=white]
    node[concept] {Beamer模板}
    [clockwise from=0]
    child[concept color=green!30!black, visible on=<{2-}>] {
      node[concept] {tikz作图}
      [clockwise from=90]
      child { node[concept] {节点} }
      child { node[concept] {边} }
      child { node[concept] {样式定义} }
      child { node[concept] {样式分配} }
    }
    child[concept color=blue!80!black, visible on=<{3-}>] {
      node[concept] {代码高亮}
      [clockwise from=-30]
      child { node[concept] {支持所有代码} }
      child { node[concept] {支持公式} }
    }
    child[concept color=red!70!structure.fg, visible on=<{4-}>] { 
      node[concept] {文献} 
      [clockwise from=-30]
      child { node[concept] {格式多样} }
      child { node[concept] {引用方便} }
    }
    child[concept color=orange!70!structure.fg, visible on=<{5-}>] { 
      node[concept] {主题} 
      [clockwise from=-90]
      child { node[concept] {主题颜色} }
      child { node[concept] {背景和前景} }
      child { node[concept] {字体} }
      child { node[concept] {图形} }
    };
\end{tikzpicture}
\visible<6->{\fadetitle{\Huge{\centerline{谢谢}}}}
\end{frame}

%------------------------------------------------

\end{document} 
